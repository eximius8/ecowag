\section{Перечень сокращений}%
\label{sec:shorthands}

\begin{tabu}to \textwidth{|X[c]|X[2]|}%
\hline%
ПДКп, мг/кг&Предельно допустимая концентрация вещества в почве\\%
\hline%
ОДК, мг/кг&Ориентировочно допустимая концентрация\\%
\hline%
ПДКв, мг/л&Предельно допустимая концентрация вещества в воде водных объектов, используемых для целей питьевого и хозяйственно{-}бытового водоснабжения\\%
\hline%
ОДУ, мг/л&Ориентировочно допустимый уровень\\%
\hline%
ОБУВ, мг/л&Ориентировочный безопасный уровень воздействия\\%
\hline%
ПДКр.х., мг/л&Предельно допустимая концентрация вещества в воде водных объектов рыбохозяйственного значения\\%
\hline%
ПДКс.с., мг/м\textsuperscript{3}&Предельно допустимая концентрация вещества среднесуточная в атмосферном воздухе населенных мест\\%
\hline%
ПДКп.п.&Предельно допустимая концентрация вещества в пищевых продуктах\\%
\hline%
ПДКм.р., мг/м\textsuperscript{3}&Предельно допустимая концентрация вещества максимально разовая в атмосферном воздухе населенных мест\\%
\hline%
ПДКр.з., мг/м\textsuperscript{3}&Предельно допустимая концентрация вещества в атмосферном воздухе рабочей зоны\\%
\hline%
МДС, мг/кг&Максимально допустимое содержание\\%
\hline%
МДУ, мг/кг&Максимально допустимый уровень\\%
\hline%
$S$, мг/л&Растворимость компонента отхода (вещества) в воде при 20°С\\%
\hline%
С\textsubscript{нас}&Насыщающая концентрация вещества в воздухе при 20°С и нормальном давлени\\%
\hline%
$K_{ow}$&Коэффициент распределения в системе октанол/вода при 20°С\\%
\hline%
LD\textsubscript{50}&Средняя смертельная доза компонента в миллиграммах действующего вещества на 1 кг живого веса, вызывающая гибель 50\% подопытных животных при однократном пероральном введении в унифицированных условиях\\%
\hline%
\end{tabu}