
Показатель $K$ степени опасности отхода для окружающей среды рассчитывается по следующей формуле:
$$
K=K_1+K_2+\dots +K_n,
$$
где $K_1, K_2, \ldots, K_n$ --- показатели степени опасности отдельных компонентов отхода для окружающей среды, 
$n$ --- количество компонентов отхода.


Отнесение отходов к классу опасности расчетным методом по показателю степени опасности отхода для окружающей 
среды осуществляется в соответствии с таблицей:

\bigskip

\noindent


\begin{tabu}to \textwidth{|X[c]|X[c]|}%
\hline%
\rowcolor{lightgray}%
\textbf{Класс опасности отхода}&\textbf{Степень опасности 	отхода для окружающей среды}\\%
\hline%
I&$10^4 \leq  K < 10^6 $\\%
\hline%
II&$10^3 \leq  K < 10^4 $\\%
\hline%
III&$10^2 \leq   K  < 10^3 $\\%
\hline%
IV&$10 < K < 10^2 $\\%
\hline%
V&$K \leq 10 $\\%
\hline
\end{tabu}
\bigskip


Степень опасности компонента отхода для окружающей среды $K_i$
рассчитывается как отношение концентрации компонента отхода $C_i$ к коэффициенту его степени опасности для окружающей среды $W_i$:
$$K_i = \frac{C_i}{W_i},$$
где $C_i$ --- концентрация $i$--тогo компонента в отходе [мг/кг]; 
         $W_i$ --- коэффициент степени опасности $i$-того компонента отхода для окружающей среды, [мг/кг].
        

Для определения коэффициента степени опасности компонента отхода для окружающей среды 
по каждому компоненту отхода устанавливаются степени их опасности для окружающей среды для различных компонентов природной среды.

